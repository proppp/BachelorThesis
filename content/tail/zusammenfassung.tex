\clearpage


\chapter{Conclusion}

The investigation of glass transition temperature (\(T_g\)) in thin polymer films has been a subject of considerable debate, particularly when comparing the behavior of these films to their bulk counterparts. This research delves into this debate by examining the complex dynamics of stiff polymers, specifically Poly-TPD and Poly-Bisphenol, under conditions of confinement. Through a focused analysis on how interface-induced changes affect these polymers, the present work contributes to a deeper understanding of the mechanisms governing the glass transition and relaxation processes in thin films.

Many studies on thin polymer films have predominantly focused on flexible polymers, which have short segmental lengths, like polystirene \cite{winkler2021}. In these polymers, the effects induced by the interface typically dissipate rapidly as one moves away from the boundary, resulting in a relatively shallow penetration depth of these effects, and thus an indifference in the $T_g$ and other properties as the thickness of the film changes. However, the present work adopts a different approach by focusing on the stiff polymers, namely Poly-TPD and P-Bisphenol. The inherent stiffness of these polymers, characterized by longer segmental lengths, enables a deeper penetration of the interface effects into the film, thus "artificially" enhancing the resolution of these effects.

By using polymers with greater chain stiffness, this research demonstrates that the effects of confinement and interfacial interactions can penetrate more deeply into the film, providing a more detailed and nuanced understanding of the dynamics at play. The findings indicate that the \(T_g\) and relaxation processes in these stiff polymer films differ markedly from those observed in thin films of shorter-segment polymers, as well as from bulk materials, underscoring the critical role of polymer chain stiffness in determining the glass transition behavior in confined systems.

The implications of this research extend beyond the specific polymers studied, highlighting the need to consider the unique dynamics of stiff polymers in confined geometries. The deeper penetration of interface-induced effects in stiff polymers like Poly-TPD and Poly-Bisphenol offers a different perspective on the glass transition and relaxation phenomena in thin films. This understanding is crucial for optimizing the performance of polymer-based thin film technologies, particularly in applications where precise control over thermal and mechanical properties is essential.


%% Lade offizielle Zusammenfassung

\null
\vspace{2\baselineskip}
\noindent
Dissertation for the attainment of the academic degree: \\
\tDegree \\
\vspace{\baselineskip}
\noindent \\
\tTitle \\
\vspace{\baselineskip}

\noindent
submitted by: \\
\tAuthor \\

\noindent
prepared in: \\
\tDepartment \\

\noindent
supervised by: \\
\tSupervisor \\

\noindent
\tSubmissionMonth{}~\tSubmissionYear \\
\vspace{5\baselineskip}


\noindent
%% Beginn der eigenen Zusammenfassung
